\documentclass[12pt]{article}
\usepackage{pmmeta}
\pmcanonicalname{TaylorFormulaRemainderVariousExpressions}
\pmcreated{2013-03-22 15:53:57}
\pmmodified{2013-03-22 15:53:57}
\pmowner{gufotta}{12050}
\pmmodifier{gufotta}{12050}
\pmtitle{Taylor formula remainder: various expressions}
\pmrecord{19}{37902}
\pmprivacy{1}
\pmauthor{gufotta}{12050}
\pmtype{Result}
\pmcomment{trigger rebuild}
\pmclassification{msc}{41A58}

% this is the default PlanetMath preamble.  as your knowledge
% of TeX increases, you will probably want to edit this, but
% it should be fine as is for beginners.

% almost certainly you want these
\usepackage{amssymb}
\usepackage{amsmath}
\usepackage{amsfonts}

% used for TeXing text within eps files
%\usepackage{psfrag}
% need this for including graphics (\includegraphics)
%\usepackage{graphicx}
% for neatly defining theorems and propositions
%\usepackage{amsthm}
% making logically defined graphics
%%%\usepackage{xypic}

% there are many more packages, add them here as you need them

% define commands here

\begin{document}
Let $f:\mathbb{R}\rightarrow \mathbb{R}$ be an $n+1$ times differentiable function, and let $%
T_{n,a}(x)$ its $n^{th\text{ }}$-degree Taylor polynomial;

Then the following expressions for the remainder $R_{n,a}(x)=f(x)-T_{n,a}(x)$
hold:

1) $$R_{n,a}(x)=\frac{1}{n!}\int_{a}^{x}f^{(n+1)}(t)(x-t)^{n}dt$$ (Integral
form)

2) $$R_{n,a}(x)=\frac{f^{(n+1)}(\eta )}{n!p}(x-\eta)^{n-p+1}(x-a)^{p}$$ for a $\eta (x)\in (a,x)$ and $\forall p>0$ (Schl\"omilch form)

3) $$R_{n,a}(x)=\frac{f^{(n+1)}(\xi )}{n!}(x-\xi )^{n}(x-a)$$ for a $\xi
(x)\in (a,x)$ \ (Cauchy form)

4) $$R_{n,a}(x)=\frac{f^{(n+1)}(\vartheta )}{(n+1)!}(x-a)^{n+1}$$ for a $%
\vartheta (x)\in (a,x)$ \ (Lagrange form)

Moreover the following result holds for the integral of the remainder from
the center point \ $a$ to an arbitrary point $b$: 

5) $$\int_{a}^{b}R_{n,a}(x)dx=\int_{a}^{b}\frac{f^{(n+1)}(x)}{(n+1)!}%
(b-x)^{n+1}dx$$

Proof:

1) Let's proceed by induction.

$n=0.$ $\int_{a}^{x}f^{^{\prime }}(t)dt=f(x)-f(a)=R_{0,a}(x)$, since $%
T_{0,a}(x)=$ $f(a)$.

Let's take it for true that $R_{n-1,a}(x)=\frac{1}{(n-1)!}%
\int_{a}^{x}f^{(n)}(t)(x-t)^{n-1}dt$,
and let's compute $\int_{a}^{x}\frac{f^{(n+1)}(t)}{n!}(x-t)^{n}dt$ by parts.
\begin{eqnarray*}
&&\int_{a}^{x}\frac{f^{(n+1)}(t)}{n!}(x-t)^{n}dt= \\
&=&\frac{f^{(n)}(t)}{n!}(x-t)^{n}|_{a}^{x}+\int_{x}^{a}\frac{f^{(n)}(t)}{n!}%
n(x-t)^{n-1}dt \\
&=&-\frac{f^{(n)}(a)}{n!}(x-a)^{n}+\int_{x}^{a}\frac{f^{(n)}(t)}{(n-1)!}%
(x-t)^{n-1}dt \\
&=&R_{n-1,a}(x)-\frac{f^{(n)}(a)}{n!}(x-a)^{n} \\
&=&f(x)-T_{n-1,a}-\frac{f^{(n)}(a)}{n!}(x-a)^{n} \\
&=&f(x)-T_{n,a}(x)=R_{n,a}(x).
\end{eqnarray*}

2) Let's write the remainder in the integral form this way:

\[
R_{n,a}(x)=\frac{1}{n!}\int_{a}^{x}f^{(n+1)}(t)(x-t)^{n}dt=\frac{1}{n!}%
\int_{a}^{x}f^{(n+1)}(t)(x-t)^{n-p+1}(x-t)^{p-1}dt
\]

Now, since $(x-t)^{p-1}$ doesn't change sign between $a$ and $x$,
we can apply the \PMlinkname{integral Mean Value theorem}{IntegralMeanValueTheorem}. So a point $\eta (x)\in
(a,x)$ exists such that
\begin{eqnarray*}
&&R_{n,a}(x)=\frac{1}{n!}\int_{a}^{x}f^{(n+1)}(t)(x-t)^{n-p+1}(x-t)^{p-1}dt
\\
&=&\frac{f^{(n+1)}(\eta )}{n!}(x-\eta )^{n-p+1}\int_{a}^{x}(x-t)^{p-1}dt \\
&=&\frac{f^{(n+1)}(\eta )}{n!p}(x-\eta )^{n-p+1}(x-a)^{p}
\end{eqnarray*}

(Note that the condition $p>0$ is needed to ensure convergence of the
integral)

3) and 4) are obtained from Schl\"omilch form by plugging in $p=1$ and $p=n+1$
respectively.

5) Let's start from the right-end side:

\begin{eqnarray*}
&&\int_{a}^{b}\frac{f^{(n+1)}(x)}{(n+1)!}(b-x)^{n+1}dx= \\
&=&\frac{f^{(n)}(x)}{(n+1)!}(b-x)^{n+1}|_{a}^{b}+\int_{a}^{b}\frac{f^{(n)}(x)%
}{(n+1)!}(n+1)(b-x)^{n}dx \\
&=&-\frac{f^{(n)}(a)}{(n+1)!}(b-a)^{n+1}+\int_{a}^{b}\frac{f^{(n)}(x)}{n!}%
(b-x)^{n}dx \\
&=&...=-\sum_{k=0}^{n}\frac{f^{(k)}(a)}{(k+1)!}(b-a)^{k+1}+\int_{a}^{b}f(x)dx
\\
&=&-\sum_{k=0}^{n}\frac{f^{(k)}(a)}{k!}\int_{a}^{b}(x-a)^{k}dx+%
\int_{a}^{b}f(x)dx \\
&=&\int_{a}^{b}\left( -\sum_{k=0}^{n}\frac{f^{(k)}(a)}{k!}%
(x-a)^{k}+f(x)\right) dx=\int_{a}^{b}R_{n,a}(x)dx.
\end{eqnarray*}

Note:

1) The proof of the integral form could also be stated as follow:

Let 

$\phi (t)=\sum_{k=0}^{n}\frac{f^{(k)}(t)}{k!}(x-t)^{k}$

Then $\phi (x)=f(x)$ and\ $\phi (a)=T_{n,a}(x)$, so that $R_{n,a}(x)=\phi
(x)-\phi (a)=\int_{a}^{x}\phi ^{\prime }(t)dt.$

Let's now compute $\phi ^{\prime }(t).$

\begin{eqnarray*}
&&\phi ^{\prime }(t)=\sum_{k=0}^{n}\frac{1}{k!}\left[
f^{(k+1)}(t)(x-t)^{k}-f^{(k)}(t)k(x-t)^{k-1}\right] \\
&=&\sum_{k=0}^{n}\frac{f^{(k+1)}(t)}{k!}(x-t)^{k}-\sum_{k=1}^{n}\frac{%
f^{(k)}(t)}{(k-1)!}(x-t)^{k-1} \\
&=&\sum_{k=0}^{n}\frac{f^{(k+1)}(t)}{k!}(x-t)^{k}-\sum_{k=0}^{n-1}\frac{%
f^{(k+1)}(t)}{k!}(x-t)^{k} \\
&=&\frac{f^{(n+1)}(t)}{n!}(x-t)^{n}.
\end{eqnarray*}

2) From the integral form of the remainder it is possible to obtain the
entire Taylor formula; indeed, repeatly integrating by parts, one gets:

\begin{eqnarray*}
&&\int_{a}^{x}\frac{f^{(n+1)}(t)}{n!}(x-t)^{n}dt=\frac{f^{(n)}(t)}{n!}%
(x-t)^{n}|_{a}^{x}+\int_{a}^{x}\frac{f^{(n)}(t)}{n!}n(x-t)^{n-1}dt \\
&=&-\frac{f^{(n)}(a)}{n!}(x-a)^{n}+\int_{a}^{x}\frac{f^{(n)}(t)}{(n-1)!}%
(x-t)^{n-1}dt \\
&=&-\frac{f^{(n)}(a)}{n!}(x-a)^{n}-\frac{f^{(n-1)}(a)}{(n-1)!}%
(x-a)^{n-1}+\int_{a}^{x}\frac{f^{(n-1)}(t)}{(n-2)!}(x-t)^{n-2}dt \\
&=&...=-\frac{f^{(n)}(a)}{n!}(x-a)^{n}-\frac{f^{(n-1)}(a)}{(n-1)!}%
(x-a)^{n-1}-...-\frac{f^{(n-k+1)}(a)}{(n-k+1)!}(x-a)^{n-k+1}+\int_{a}^{x}%
\frac{f^{(n-k+1)}(t)}{(n-k)!}(x-t)^{n-k}dt \\
&&... \\
&=&-\sum_{k=1}^{n}\frac{f^{(k)}(a)}{k!}(x-a)^{k}+\int_{a}^{x}f^{\prime }(t)dt
\\
&=&-\sum_{k=1}^{n}\frac{f^{(k)}(a)}{k!}(x-a)^{k}+f(x)-f(a) \\
&=&-\sum_{k=0}^{n}\frac{f^{(k)}(a)}{k!}(x-a)^{k}+f(x).
\end{eqnarray*}

that is

\begin{eqnarray*}
f(x) &=&\sum_{k=0}^{n}\frac{f^{(k)}(a)}{k!}(x-a)^{k}+\int_{a}^{x}\frac{%
f^{(n+1)}(t)}{n!}(x-t)^{n}dt \\
&=&\sum_{k=0}^{n}\frac{f^{(k)}(a)}{k!}(x-a)^{k}+R_{n,a}(x).
\end{eqnarray*}
%%%%%
%%%%%
\end{document}
