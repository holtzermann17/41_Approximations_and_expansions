\documentclass[12pt]{article}
\usepackage{pmmeta}
\pmcanonicalname{TaylorSeriesDerivationOf}
\pmcreated{2013-03-22 15:25:24}
\pmmodified{2013-03-22 15:25:24}
\pmowner{apmc}{9183}
\pmmodifier{apmc}{9183}
\pmtitle{Taylor series, derivation of}
\pmrecord{6}{37268}
\pmprivacy{1}
\pmauthor{apmc}{9183}
\pmtype{Derivation}
\pmcomment{trigger rebuild}
\pmclassification{msc}{41A58}

% this is the default PlanetMath preamble.  as your knowledge
% of TeX increases, you will probably want to edit this, but
% it should be fine as is for beginners.

% almost certainly you want these
\usepackage{amssymb}
\usepackage{amsmath}
\usepackage{amsfonts}

% used for TeXing text within eps files
%\usepackage{psfrag}
% need this for including graphics (\includegraphics)
%\usepackage{graphicx}
% for neatly defining theorems and propositions
%\usepackage{amsthm}
% making logically defined graphics
%%%\usepackage{xypic}

% there are many more packages, add them here as you need them

% define commands here
\begin{document}
Let $f(x)$ be given by the following power series:

\begin{center}$\displaystyle f(x)=c_0+c_1(x-a)+c_2(x-a)^2+\cdots+c_n(x-a)^n+\cdots=\sum_{k=0}^{\infty}c_k(x-a)^k$\end{center}

Now let's compute a few derivatives at $x=a.$

\begin{center}$\displaystyle f(a)=c_0;\quad f'(a)=c_1;\quad f''(a)=2c_2;\quad f^{(3)}(a)=6c_3;\quad f^{(n)}(a)=n!c_n$\end{center}

From this, it is clear that $\displaystyle c_n=\frac{f^{(n)}(a)}{n!}$, thus the series can be written as:

\begin{center}$\displaystyle T_n=\sum_{k=0}^nc_k(x-a)^k=\sum_{k=0}^n\frac{f^{(k)}(a)}{k!}(x-a)^k$\end{center}

where $f(x)=T_{\infty}$.
%%%%%
%%%%%
\end{document}
