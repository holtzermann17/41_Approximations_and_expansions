\documentclass[12pt]{article}
\usepackage{pmmeta}
\pmcanonicalname{ApproximationOfTheLogFunction}
\pmcreated{2013-03-22 15:18:38}
\pmmodified{2013-03-22 15:18:38}
\pmowner{Mathprof}{13753}
\pmmodifier{Mathprof}{13753}
\pmtitle{approximation of the log function}
\pmrecord{10}{37113}
\pmprivacy{1}
\pmauthor{Mathprof}{13753}
\pmtype{Derivation}
\pmcomment{trigger rebuild}
\pmclassification{msc}{41A60}

\usepackage{amssymb}
\usepackage{amsmath}
\usepackage{amsfonts}

\begin{document}
Because
\begin{eqnarray*}
\lim_{x\rightarrow\ 0} x \log\left(x\right) &=& \lim_{x\rightarrow 0}  x^x - 1
\end{eqnarray*}
we can approximate $\log{\left(x\right)}$ for small $x$:
\begin{eqnarray*}
\log\left(x\right) &\approx& \frac{ x^x - 1 }{ x }.
\end{eqnarray*}

A perhaps more interesting and useful result is that for $x$ small we have the approximation
\[ \log{(1+x)} \approx x. \]

In general, if $x$ is smaller than $0.1$ our approximation is practical.  This occurs because for small $x$, the area under the curve (which is what $\log$ is a measurement of) is approximately that of a rectangle of height 1 and width $x$.

Now when we combine this approximation with the formula $\log(ab)=\log(a) + \log(b)$, we can now approximate the logarithm of many positive numbers.  In fact, scientific calculators use a (somewhat more precise) version of the same procedure.

For example, suppose we wanted $\log(1.21)$.  If we estimate $\log(1.1) + \log(1.1)$ by taking $0.1 + 0.1 = 0.2$, we would be pretty close.
%%%%%
%%%%%
\end{document}
