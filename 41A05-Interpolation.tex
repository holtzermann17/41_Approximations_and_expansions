\documentclass[12pt]{article}
\usepackage{pmmeta}
\pmcanonicalname{Interpolation}
\pmcreated{2013-03-22 14:20:05}
\pmmodified{2013-03-22 14:20:05}
\pmowner{CWoo}{3771}
\pmmodifier{CWoo}{3771}
\pmtitle{interpolation}
\pmrecord{13}{35805}
\pmprivacy{1}
\pmauthor{CWoo}{3771}
\pmtype{Definition}
\pmcomment{trigger rebuild}
\pmclassification{msc}{41A05}
\pmclassification{msc}{65D05}
\pmdefines{breakpoints}
\pmdefines{interpolating function}

\endmetadata

% this is the default PlanetMath preamble.  as your knowledge
% of TeX increases, you will probably want to edit this, but
% it should be fine as is for beginners.

% almost certainly you want these
\usepackage{amssymb}
\usepackage{amsmath}
\usepackage{amsfonts}

% used for TeXing text within eps files
%\usepackage{psfrag}
% need this for including graphics (\includegraphics)
%\usepackage{graphicx}
% for neatly defining theorems and propositions
%\usepackage{amsthm}
% making logically defined graphics
%%%\usepackage{xypic}

% there are many more packages, add them here as you need them

% define commands here
\begin{document}
\emph{Interpolation} is a set of techniques in approximation where, given a set of paired data points
$$(x_1,y_1),(x_2,y_2),\ldots, (x_n,y_n),\ldots$$ one is often interested in 
\begin{itemize}
\item finding a relation (usually in the form of a function $f$) that passes through (or is satisfied by) every one of these points, if the relation is unknown at the beginning,
\item finding a simplified relation to replace the original known relation that is very complicated and difficult to use,
\item finding other paired data points $(x_{\alpha},y_{\alpha})$ in addition to the existing ones.
\end{itemize}
The data points $(x_i,y_i)$ are called the \emph{breakpoints}, and the function $f$ is the \emph{interpolating function} such that $f(x_i)=y_i$ for each $i$.

The choice of the interpolating function depends on what we wish to do with it. In some cases a polynomial is required, sometimes a piecewise linear  function is prefered (linear interpolation), other times a \PMlinkid{spline}{4339} is of interest, when the interpolating function is required to not only to be continuous, but differentiable, or even smooth.

Even different strategies for finding the same interpolating function are of interest. The Lagrange interpolation formula is a direct way to calculate the interpolating polynomial. The Vandermonde interpolation formula is mainly of interest as a theoretical tool. Numerical implementation of Vandermonde interpolation involves solution of large ill conditioned linear systems, so numerical stability is questionable.
%%%%%
%%%%%
\end{document}
